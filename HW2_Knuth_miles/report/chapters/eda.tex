\section{Geographic Distribution Analysis}
The geographic distribution of cities in the Knuth Miles dataset reveals several interesting patterns. Figure \ref{fig:geo_dist} shows the spatial distribution of cities, with point sizes proportional to population and colors indicating population density.

\begin{figure}[H]
    \centering
    \includegraphics[width=0.8\textwidth]{figures/geo_distribution.png}
    \caption{Geographic distribution of cities, with point sizes proportional to population}
    \label{fig:geo_dist}
\end{figure}

Key observations from the geographic analysis include:
\begin{itemize}
    \item Concentration of major cities along the East Coast
    \item Sparse distribution in the central and western regions
    \item Notable clustering around major metropolitan areas
    \item Clear separation between US and Canadian cities
\end{itemize}

\section{Population Analysis}
The population distribution across cities shows a highly skewed pattern, typical of urban systems. Figure \ref{fig:pop_dist} illustrates both the raw and log-transformed population distributions.

\begin{figure}[H]
    \centering
    \begin{subfigure}[b]{0.48\textwidth}
        \includegraphics[width=\textwidth]{figures/pop_distribution.png}
        \caption{Raw population distribution}
    \end{subfigure}
    \begin{subfigure}[b]{0.48\textwidth}
        \includegraphics[width=\textwidth]{figures/pop_log_distribution.png}
        \caption{Log-transformed distribution}
    \end{subfigure}
    \caption{Population distribution analysis}
    \label{fig:pop_dist}
\end{figure}

Population statistics:
\begin{itemize}
    \item Mean population: [Value]
    \item Median population: [Value]
    \item Standard deviation: [Value]
    \item Minimum population: [Value]
    \item Maximum population: [Value]
\end{itemize}

The log-normal distribution of populations suggests a hierarchical urban system, consistent with urban scaling theory.

\section{Network Structure Analysis}
The transportation network exhibits several interesting structural properties. Figure \ref{fig:network} shows the network visualization with communities highlighted.

\begin{figure}[H]
    \centering
    \includegraphics[width=0.8\textwidth]{figures/network_communities.png}
    \caption{Network visualization with detected communities}
    \label{fig:network}
\end{figure}

Key network metrics:
\begin{itemize}
    \item Number of nodes: 128
    \item Number of edges: [Value]
    \item Average degree: [Value]
    \item Network density: [Value]
    \item Average clustering coefficient: [Value]
    \item Average shortest path length: [Value]
\end{itemize}

\section{Centrality Analysis}
The centrality analysis reveals the most important cities in the transportation network. Figure \ref{fig:centrality} shows the distribution of different centrality measures.

\begin{figure}[H]
    \centering
    \begin{subfigure}[b]{0.32\textwidth}
        \includegraphics[width=\textwidth]{figures/degree_centrality.png}
        \caption{Degree centrality}
    \end{subfigure}
    \begin{subfigure}[b]{0.32\textwidth}
        \includegraphics[width=\textwidth]{figures/betweenness_centrality.png}
        \caption{Betweenness centrality}
    \end{subfigure}
    \begin{subfigure}[b]{0.32\textwidth}
        \includegraphics[width=\textwidth]{figures/closeness_centrality.png}
        \caption{Closeness centrality}
    \end{subfigure}
    \caption{Distribution of centrality measures}
    \label{fig:centrality}
\end{figure}

Top 5 cities by each centrality measure:
\begin{itemize}
    \item Degree Centrality: [List]
    \item Betweenness Centrality: [List]
    \item Closeness Centrality: [List]
\end{itemize}

\section{Community Detection}
The Louvain community detection algorithm identified [Number] distinct communities in the network. Figure \ref{fig:communities} shows the geographic distribution of these communities.

\begin{figure}[H]
    \centering
    \includegraphics[width=0.8\textwidth]{figures/community_map.png}
    \caption{Geographic distribution of detected communities}
    \label{fig:communities}
\end{figure}

Key findings from community detection:
\begin{itemize}
    \item Communities often align with geographic regions
    \item Some communities cross state/provincial boundaries
    \item Major metropolitan areas often form their own communities
    \item Clear separation between US and Canadian communities
\end{itemize}

This exploratory analysis provides a foundation for more detailed investigation of the network's properties and their implications for transportation planning and urban development. 