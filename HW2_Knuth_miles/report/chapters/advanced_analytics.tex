\section{Network Robustness Analysis}
The transportation network's resilience to disruptions was analyzed through systematic node and edge removal simulations. Figure \ref{fig:robustness} shows the impact of targeted and random attacks on network connectivity.

\begin{figure}[H]
    \centering
    \includegraphics[width=0.8\textwidth]{figures/network_robustness.png}
    \caption{Network robustness under different attack scenarios}
    \label{fig:robustness}
\end{figure}

Key findings:
\begin{itemize}
    \item Network shows high resilience to random failures
    \item Targeted attacks on high-centrality nodes cause significant disruption
    \item Critical threshold for network breakdown: [Value]
    \item Most vulnerable regions: [List]
\end{itemize}

\section{Spatial Autocorrelation Analysis}
The spatial distribution of network properties was analyzed using Moran's I statistic. Figure \ref{fig:spatial_autocorr} shows the spatial autocorrelation patterns for different network metrics.

\begin{figure}[H]
    \centering
    \includegraphics[width=0.8\textwidth]{figures/spatial_autocorrelation.png}
    \caption{Spatial autocorrelation of network properties}
    \label{fig:spatial_autocorr}
\end{figure}

Results:
\begin{itemize}
    \item Strong positive spatial autocorrelation in centrality measures
    \item Regional clustering of similar network properties
    \item Significant spatial patterns in community structure
    \item Geographic barriers affecting network connectivity
\end{itemize}

\section{Hierarchical Structure Analysis}
The network's hierarchical organization was analyzed using the k-core decomposition method. Figure \ref{fig:hierarchy} illustrates the hierarchical levels and their properties.

\begin{figure}[H]
    \centering
    \includegraphics[width=0.8\textwidth]{figures/network_hierarchy.png}
    \caption{Network hierarchical structure}
    \label{fig:hierarchy}
\end{figure}

Hierarchical properties:
\begin{itemize}
    \item Number of hierarchical levels: [Value]
    \item Core-periphery structure strength: [Value]
    \item Key cities at each hierarchical level: [List]
    \item Regional hierarchical patterns: [Description]
\end{itemize}

\section{Temporal Evolution Analysis}
While the dataset represents a static snapshot, we can infer potential evolutionary patterns through structural analysis. Figure \ref{fig:evolution} shows the inferred growth patterns.

\begin{figure}[H]
    \centering
    \includegraphics[width=0.8\textwidth]{figures/network_evolution.png}
    \caption{Inferred network evolution patterns}
    \label{fig:evolution}
\end{figure}

Evolutionary insights:
\begin{itemize}
    \item Preferential attachment patterns in network growth
    \item Historical development of major hubs
    \item Regional development trajectories
    \item Future growth potential areas
\end{itemize}

\section{Multivariate Analysis}
A comprehensive analysis of the relationships between different network properties was conducted. Figure \ref{fig:multivariate} shows the correlation patterns between key metrics.

\begin{figure}[H]
    \centering
    \includegraphics[width=0.8\textwidth]{figures/multivariate_analysis.png}
    \caption{Correlation analysis of network properties}
    \label{fig:multivariate}
\end{figure}

Key relationships:
\begin{itemize}
    \item Population size vs. network centrality
    \item Geographic location vs. community membership
    \item Distance vs. connection strength
    \item Economic indicators vs. network position
\end{itemize}

\section{Policy Implications}
The advanced analysis reveals several important implications for transportation and urban planning:

\begin{itemize}
    \item Critical infrastructure protection priorities
    \item Regional development opportunities
    \item Transportation network optimization strategies
    \item Urban growth management recommendations
\end{itemize}

These findings provide a robust foundation for evidence-based policy decisions and future research directions in urban transportation network analysis. 